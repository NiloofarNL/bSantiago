% Options for packages loaded elsewhere
\PassOptionsToPackage{unicode}{hyperref}
\PassOptionsToPackage{hyphens}{url}
%
\documentclass[
]{article}
\usepackage{amsmath,amssymb}
\usepackage{iftex}
\ifPDFTeX
  \usepackage[T1]{fontenc}
  \usepackage[utf8]{inputenc}
  \usepackage{textcomp} % provide euro and other symbols
\else % if luatex or xetex
  \usepackage{unicode-math} % this also loads fontspec
  \defaultfontfeatures{Scale=MatchLowercase}
  \defaultfontfeatures[\rmfamily]{Ligatures=TeX,Scale=1}
\fi
\usepackage{lmodern}
\ifPDFTeX\else
  % xetex/luatex font selection
\fi
% Use upquote if available, for straight quotes in verbatim environments
\IfFileExists{upquote.sty}{\usepackage{upquote}}{}
\IfFileExists{microtype.sty}{% use microtype if available
  \usepackage[]{microtype}
  \UseMicrotypeSet[protrusion]{basicmath} % disable protrusion for tt fonts
}{}
\makeatletter
\@ifundefined{KOMAClassName}{% if non-KOMA class
  \IfFileExists{parskip.sty}{%
    \usepackage{parskip}
  }{% else
    \setlength{\parindent}{0pt}
    \setlength{\parskip}{6pt plus 2pt minus 1pt}}
}{% if KOMA class
  \KOMAoptions{parskip=half}}
\makeatother
\usepackage{xcolor}
\usepackage[margin=1in]{geometry}
\usepackage{longtable,booktabs,array}
\usepackage{calc} % for calculating minipage widths
% Correct order of tables after \paragraph or \subparagraph
\usepackage{etoolbox}
\makeatletter
\patchcmd\longtable{\par}{\if@noskipsec\mbox{}\fi\par}{}{}
\makeatother
% Allow footnotes in longtable head/foot
\IfFileExists{footnotehyper.sty}{\usepackage{footnotehyper}}{\usepackage{footnote}}
\makesavenoteenv{longtable}
\usepackage{graphicx}
\makeatletter
\def\maxwidth{\ifdim\Gin@nat@width>\linewidth\linewidth\else\Gin@nat@width\fi}
\def\maxheight{\ifdim\Gin@nat@height>\textheight\textheight\else\Gin@nat@height\fi}
\makeatother
% Scale images if necessary, so that they will not overflow the page
% margins by default, and it is still possible to overwrite the defaults
% using explicit options in \includegraphics[width, height, ...]{}
\setkeys{Gin}{width=\maxwidth,height=\maxheight,keepaspectratio}
% Set default figure placement to htbp
\makeatletter
\def\fps@figure{htbp}
\makeatother
\setlength{\emergencystretch}{3em} % prevent overfull lines
\providecommand{\tightlist}{%
  \setlength{\itemsep}{0pt}\setlength{\parskip}{0pt}}
\setcounter{secnumdepth}{-\maxdimen} % remove section numbering
\newlength{\cslhangindent}
\setlength{\cslhangindent}{1.5em}
\newlength{\csllabelwidth}
\setlength{\csllabelwidth}{3em}
\newlength{\cslentryspacingunit} % times entry-spacing
\setlength{\cslentryspacingunit}{\parskip}
\newenvironment{CSLReferences}[2] % #1 hanging-ident, #2 entry spacing
 {% don't indent paragraphs
  \setlength{\parindent}{0pt}
  % turn on hanging indent if param 1 is 1
  \ifodd #1
  \let\oldpar\par
  \def\par{\hangindent=\cslhangindent\oldpar}
  \fi
  % set entry spacing
  \setlength{\parskip}{#2\cslentryspacingunit}
 }%
 {}
\usepackage{calc}
\newcommand{\CSLBlock}[1]{#1\hfill\break}
\newcommand{\CSLLeftMargin}[1]{\parbox[t]{\csllabelwidth}{#1}}
\newcommand{\CSLRightInline}[1]{\parbox[t]{\linewidth - \csllabelwidth}{#1}\break}
\newcommand{\CSLIndent}[1]{\hspace{\cslhangindent}#1}
\ifLuaTeX
  \usepackage{selnolig}  % disable illegal ligatures
\fi
\IfFileExists{bookmark.sty}{\usepackage{bookmark}}{\usepackage{hyperref}}
\IfFileExists{xurl.sty}{\usepackage{xurl}}{} % add URL line breaks if available
\urlstyle{same}
\hypersetup{
  pdftitle={Research paper},
  pdfauthor={Niloofar},
  hidelinks,
  pdfcreator={LaTeX via pandoc}}

\title{Research paper}
\author{Niloofar}
\date{2024-06-07}

\begin{document}
\maketitle

\#A Dataset to Study Transportation, Residential Context, and Well-being
in Santiago, Chile

\#\#Abstract

Nowadays commuting as a daily travel mostly between work and home is
considered as an inevitable part of modern lifestyle. Understanding
commuting patterns and travel behavior is important for analyzing
stress-related issues, consequences and coping strategies. This dataset
is part of a thesis aiming to identify how motorized and non-motorized
commuters interact with experiencing stress during their travels. The
dataset is based on a paper-based survey conducted face-to-face during
November and December 2016 in Santiago, Chile in 2016. A large body of
data has made inroads investigating the psychological impact on
travellers ranging from positive feelings of enjoyment in some to the
sensation of stress in many others that can affect the effectiveness of
policy measures and are known to affect health outcomes. The data set
contributes to the psychological impact on travellers on both active and
motorized modes of transportation to examine not only the feeling of
stress but also how these effects are experienced by travellers and
investigate the importance that travellers attach to their feelings of
stress which makes it valuable for researchers who focused on public
sector development and health-related policies. Furthermore, the dataset
lies in its potential contributions to transportation policies, public
sector development, and health-related policies, as it addresses
commuting stress and diverse travel-related issues, encouraging further
exploration and fostering reproducibility through transparent
methodologies, comprehensive documentation, and open data practices,
thereby laying a foundation for future research.

\hypertarget{introduction}{%
\subsection{Introduction}\label{introduction}}

Understanding the multi-dimensional nature of urban living in Santiago
is crucial for effective planning and policymaking. As Gainza and Livert
(2013) have mentioned that travel has influenced the urban form of
Santiago as the city spreads out because of the monocentric nature of
that. In terms of urban development and sprawling land uses, Hall, Hall,
Zegras, and Rojas (1994) indicated that Santiago has experienced highly
deregulated urban growth on the urban periphery at the expense of the
conventional business district and more central comunas. Furthermore,
Basso et al. (2020) focused on public transport accessibility and its
role in mitigating inequalities across Santiago. Living and traveling in
Santiago can be influenced by social exclusion and income level as Ureta
(2008) analyzed the mobility patterns of low-income people in everyday
life and their situation of social exclusion.

As Graells-Garrido, Peredo, and Garcı́a (2016) mentioned, urban policy
takes several inputs into account. Travel surveys are one of them,
because they provide rich information about the city: where people
travel to (and from), the purpose of the trip (e.g., commuting to work),
the mode (e.g., metro), the time spent traveling, as well as other
socio-demographic variables. Large-scale travel surveys are invaluable
sources of information to understand travel behaviour and other aspects
of the urban experience, such as the residential context. By their
nature, they often shy away from overloading respondents with additional
questions.

This paper introduces a comprehensive dataset derived from a
meticulously designed survey that captures various aspects of the
experience of living and moving in a major city in the Global South. The
dataset includes self- assessed health questions, respondents' feelings
and emotions during commuting experience, characteristics of the built
environment, individual demographics, commuting behaviors, the social
experience of a variety of transportation modes, and patterns of use of
information and telecommunication technologies. Moreover, it encompasses
respondents' attitudes towards the effectiveness of the local
transportation system, as well as their perceptions of the city's
sustainability and natural environment.

The combination of all these aspects makes this data set a novel,
multifaceted and significant as it allows for a holistic analysis of how
different modes of transportation influence daily life and individual
well-being in an urban setting. By integrating traditional urban study
variables with emerging factors like technology use and environmental
attitudes, this dataset provides a unique lens through which the urban
experience in Santiago can be understood. Furthermore, the inclusion of
geographical variables offers critical insights into the spatial
patterns of urban living, highlighting how location influences lifestyle
and transportation choices in Santiago.

\hypertarget{data-collection-materials-and-methods}{%
\subsection{Data Collection, Materials and
Methods}\label{data-collection-materials-and-methods}}

The study is based on a paper-based survey conducted face-to-face in
Santiago during November and December 2016 by Dr.~Beatriz Mella-Lira.
The survey collected information on a wide range of travel- related
issues. The data collection considered a quota-sampling method based on
the information from the Pre-census of 2012 and with a total of 451
persons. More information about data specifications can be seen in
Table\_1.

\begin{longtable}[]{@{}
  >{\raggedright\arraybackslash}p{(\columnwidth - 2\tabcolsep) * \real{0.6000}}
  >{\raggedright\arraybackslash}p{(\columnwidth - 2\tabcolsep) * \real{0.4000}}@{}}
\caption{Specifications Table}\tabularnewline
\toprule\noalign{}
\begin{minipage}[b]{\linewidth}\raggedright
Items
\end{minipage} & \begin{minipage}[b]{\linewidth}\raggedright
Explanation
\end{minipage} \\
\midrule\noalign{}
\endfirsthead
\toprule\noalign{}
\begin{minipage}[b]{\linewidth}\raggedright
Items
\end{minipage} & \begin{minipage}[b]{\linewidth}\raggedright
Explanation
\end{minipage} \\
\midrule\noalign{}
\endhead
\bottomrule\noalign{}
\endlastfoot
Subject area & Transportation, Geography, Public Health and Health
Policy, Urban development \\
More specific subject area & Transport inequalities, Stress and limited
horizons, Travel behaviour, Global South \\
Type of data & \texttt{R} Data Package \\
How data was acquired & The survey was conducted using a questionnaire.
The instrument contains some quantitative variables regarding the
individual characteristics of respondents and mostly 5-point Likert
scale responses in the rest of the questionnaire \\
Data format & Thematic tables and documentation in native \texttt{R}
format. The thematic tables can be linked by means of a common
\texttt{ID} field \\
Parameters for data collection & The survey was collected using a
quota-sampling method based on the information from the Pre-census of
2012, and in total, 451 persons validly completed the face-to-face
survey in Santiago, Chile in 2016. The survey collected information on a
wide range of travel-related issues (socio-demographics, health-related,
perceptions and travel behaviour, travel choices and planning, social
interaction factors, built environment, and so on) \\
Description of data collection & Data was acquired through the 5-Likert
scale questionnaire regarding most sections of the questionnaire, using
a face-to-face and quota-sampling method for individual
characteristics \\
Data source location & Santiago, Chile \\
Data accessibility & \url{https://paezha.github.io/bSantiago/} \\
\end{longtable}

\hypertarget{how-does-data-preprocessing-contribute-to-reproducibility}{%
\subsection{How does data preprocessing contribute to
reproducibility?}\label{how-does-data-preprocessing-contribute-to-reproducibility}}

As mentioned earlier, the current study is based on a survey. In the
data preprocessing step we went through addressing some issues such as
splitting multiple answers in a single cell of the survey. Having the
answers separated into distinct columns, allows future researchers to
gain flexibility in analyzing the data and generating tables tailored to
their research questions or goals. The specific analyses and tables
created will depend on the nature of the data and the research
objectives.

When starting from scratch after splitting and reshaping the data,
researchers can use various techniques for a more granular analysis of
individual responses and choosing the specific answer among multiple
ones in a single cell.

One idea is using frequency tables as they can observe the occurrences
of each response for each split variable useful in understanding the
distribution of the response.

The second approach way is to take advantage of cross-tabulations for
split variables against each other, as it gives insights into
relationships and patterns in the data.

Also calculating summary statistics for each split variable allows us to
understand central tendency and variability. There are many other ways,
such as visualization, comparisons, tests, and data exploration, to
specify the appropriate answer out of multiple ones.

When starting from the raw data and following the same preprocessing and
analysis steps, researchers could generally end up with similar tables
and results. However, several factors can influence whether the results
are exactly the same such as differences in data cleaning and handling
missing data, parameters or thresholds according to the analysis, and
random sampling (if applicable).

In summary, while exact replication is not always guaranteed, careful
documentation, consistency in preprocessing, and transparent reporting
of analysis steps contribute to the reproducibility of results.
Additionally, using the bSantiago GitHub repository and shared codes can
facilitate collaboration and make it easier for future researchers to
reproduce the results.

\hypertarget{brief-overview-of-the-tables-in-the-data-package}{%
\subsection{Brief overview of the tables in the data
Package}\label{brief-overview-of-the-tables-in-the-data-package}}

The dataset contains multiple tables, each focusing on different aspects
of transportation, residential context, and well-being of urban
residents in Santiago. Below is a summarized overview of these tables:

Santiago\_IC (Individual Characteristics): This section includes a wide
range of socio-economic and demographic attributes such as age, gender,
education, occupation, and household context.

Santiago\_TW (Commuting Behavior): Commuting behavior and job-related
variables including job accessibility, job satisfaction, commute
duration, and transportation expenditure can be found in this section.
In terms of the level of satisfaction with their current job, statistics
shows that almost 60\% of respondents are at least highly satisfied. We
also see that long commutes are frequent in this sample, with about
one-third of respondents spending 1 h or more travelling and about
one-quarter of respondents spending between 40 minutes and one hour in
their daily commute. This distribution is noteworthy because time spent
commuting has been recognized as a factor that can affect physical and
mental health and well-being in particular in association with motorized
transportation (Brutus, Javadian, and Panaccio 2017).

Santiago\_SI (Social Interaction): This section is focused on social
interaction variables such as: levels of social interaction and
discrimination experienced during commuting, including interaction
levels and discrimination by mode of transport.

Social interaction is a topic of interest for mode-related choices given
earlier evidence that for some commuters privacy is an important
consideration and/or a way to manage social stressors (Páez and Whalen
2010); (Lowe and Mosby 2016); (Gardner and Abraham 2007).

Santiago\_IS (Information and Telecommunication, Mode Shifting):
Variables related to use of technology for travel information, quality
of mode shifts, waiting times, travel times, and difficulty in mode
changing can be found in this section.

Santiago\_BE (Built Environment): Perceptions and importance of built
environment attributes such as space for autos, parking, highways,
pedestrian space, and cleanliness of bus stops are included in
Santiago\_BE section.

Santiago\_H (Health): Health-related information including stress,
physical effort, proximity to other travelers, environmental pollution,
safety, and comfort during commutes have been stored in health section.

Santiago\_FE (Feelings and Emotions): Emotional responses and feelings
associated with different modes of transportation, including freedom,
safety, luxury, environmental care, and social interaction can be found
in feeling and emotions section.

Santiago\_RPD (Decision-making and Planning): Variables related to
decision-making and planning, are namely as: accessibility to
employment, social and recreational activities, importance of
transportation options, and quality of public transport.

Santiago\_NS (Nature and Sustainability): Perspectives on sustainable
transportation, willingness to pay for sustainable modes, and the
importance of environmental attributes like trees and parks are grouped
as nature and sustainability section.

comuna\_1 (Geographical Variables): Geographical information related to
the respondents' location, including commune and macro zone identifiers,
providing region-specific insights can be found in comuna section.

Each table is linked by a common unique identifier (ID), allowing for
multi-theme analysis of the data. The dataset provides a comprehensive
view of how transportation and built environment influence daily life
and well-being in Santiago.

\hypertarget{discussion-and-conclusion}{%
\subsection{Discussion and Conclusion}\label{discussion-and-conclusion}}

The current summary of the Santiago dataset showcases a comprehensive
commitment to ensuring reproducibility through detailed stages including
data collection, cleaning, preprocessing, documentation, storage, and
version control. The inclusion of a specifications table in the
documentation provides a quick and concise reference, summarizing key
details such as the subject area, data format, collection parameters,
and accessibility. This not only aids in reproducibility by offering a
high-level overview of the dataset's characteristics but also assists
researchers in quickly understanding the pertinent details of the
dataset.

The data collection process, employing a quota-sampling method, is
crucial for the dataset's reproducibility. The detailed explanation of
how the sample was obtained allows future researchers to understand the
representativeness of the dataset in relation to the population of
Santiago in 2016. This level of transparency makes it feasible to
replicate the sampling method in future studies or to compare it with
alternative strategies, thus contributing to the reproducibility of the
research.

Additionally, the thorough description of the paper-based survey,
including its content focusing on areas such as demographics, health,
and emotional states, enhances the ability to reproduce the study.
Researchers aiming to replicate or extend this work can gain valuable
insights into the survey structure, question types, and the scope of
topics covered, ensuring that future studies can align with or build
upon the original research framework. The data cleaning and
preprocessing steps are meticulously documented, including the tools and
packages used, such as the dplyr package in R. This documentation
provides a clear roadmap for future researchers, ensuring that the
processed dataset meets the quality standards intended by the original
researchers. Moreover, the use of a Quarto Markdown file for data
documentation helps in providing a clear and concise reference for
understanding the dataset's structure and content, including informative
tables and graphs generated with various R packages. Making the dataset
publicly available on a dedicated GitHub repository enhances
reproducibility by fostering transparency and openness. The use of
version control with Git ensures that any changes to the dataset are
meticulously tracked over time, allowing for a clear understanding of
its evolution and facilitating reproducibility in subsequent analyses.

In summary, the Santiago dataset is not just a collection of data but a
well-documented and robustly managed resource that encourages
reproducibility through transparent methodologies, comprehensive
documentation, and open data practices. This approach not only secures
the reliability of the current study but also sets a strong foundation
for future research endeavors and collaborations within the scientific
community.

\hypertarget{acknowledgements}{%
\subsection{Acknowledgements}\label{acknowledgements}}

Profound gratitude is extended to the esteemed supervisor, Dr.~Antonio
Páez, for invaluable guidance and support throughout the research
endeavor and to Dr.~Beatriz Mella-Lira for generous support and
provision of survey materials.

\hypertarget{declaration-of-conflicting-interests}{%
\subsection{Declaration of conflicting
interests}\label{declaration-of-conflicting-interests}}

The author(s) declared no potential conflicts of interest with respect
to the research, authorship, and/or publication of this article.

\hypertarget{funding}{%
\subsection{Funding}\label{funding}}

This research has been supported through a M.Sc. research grant of the
McMaster university.

\hypertarget{footnotes}{%
\subsection{Footnotes}\label{footnotes}}

GitHub repository: \href{https://paezha.github.io/bSantiago/}{bSantiago}

\hypertarget{orcid-ids}{%
\subsection{ORCID IDS}\label{orcid-ids}}

\href{https://orcid.org/0000-0003-4344-0412}{Niloofar Nalaee}

\hypertarget{refs}{}
\begin{CSLReferences}{1}{0}
\leavevmode\vadjust pre{\hypertarget{ref-basso2020accessibility}{}}%
Basso, Franco, Jonathan Frez, Luis Martı́nez, Raúl Pezoa, and Mauricio
Varas. 2020. {``Accessibility to Opportunities Based on Public Transport
Gps-Monitored Data: The Case of Santiago, Chile.''} \emph{Travel
Behaviour and Society} 21: 140--53.

\leavevmode\vadjust pre{\hypertarget{ref-brutus2017cycling}{}}%
Brutus, Stéphane, Roshan Javadian, and Alexandra Joelle Panaccio. 2017.
{``Cycling, Car, or Public Transit: A Study of Stress and Mood Upon
Arrival at Work.''} \emph{International Journal of Workplace Health
Management} 10 (1): 13--24.

\leavevmode\vadjust pre{\hypertarget{ref-gainza2013urban}{}}%
Gainza, Xabier, and Felipe Livert. 2013. {``Urban Form and the
Environmental Impact of Commuting in a Segregated City, Santiago de
Chile.''} \emph{Environment and Planning B: Planning and Design} 40 (3):
507--22.

\leavevmode\vadjust pre{\hypertarget{ref-gardner2007drives}{}}%
Gardner, Benjamin, and Charles Abraham. 2007. {``What Drives Car Use? A
Grounded Theory Analysis of Commuters' Reasons for Driving.''}
\emph{Transportation Research Part F: Traffic Psychology and Behaviour}
10 (3): 187--200.

\leavevmode\vadjust pre{\hypertarget{ref-graells2016sensing}{}}%
Graells-Garrido, Eduardo, Oscar Peredo, and José Garcı́a. 2016.
{``Sensing Urban Patterns with Antenna Mappings: The Case of Santiago,
Chile.''} \emph{Sensors} 16 (7): 1098.

\leavevmode\vadjust pre{\hypertarget{ref-hall1994transportation}{}}%
Hall, Stephen, Christopher Zegras, and Henry Malbrán Rojas. 1994.
{``Transportation and Energy in Santiago, Chile.''} \emph{Transport
Policy} 1 (4): 233--43.

\leavevmode\vadjust pre{\hypertarget{ref-lowe2016conceptual}{}}%
Lowe, Kate, and Kim Mosby. 2016. {``The Conceptual Mismatch: A
Qualitative Analysis of Transportation Costs and Stressors for
Low-Income Adults.''} \emph{Transport Policy} 49: 1--8.

\leavevmode\vadjust pre{\hypertarget{ref-paez2010enjoyment}{}}%
Páez, Antonio, and Kate Whalen. 2010. {``Enjoyment of Commute: A
Comparison of Different Transportation Modes.''} \emph{Transportation
Research Part A: Policy and Practice} 44 (7): 537--49.

\leavevmode\vadjust pre{\hypertarget{ref-ureta2008move}{}}%
Ureta, Sebastian. 2008. {``To Move or Not to Move? Social Exclusion,
Accessibility and Daily Mobility Among the Low-Income Population in
Santiago, Chile.''} \emph{Mobilities} 3 (2): 269--89.

\end{CSLReferences}

\end{document}
