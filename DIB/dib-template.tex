%% This is file `dib-template.tex',
%%
%% Copyright 2020 Elsevier Ltd
%%
%% This file is part of the 'Elsarticle Bundle'.
%% ---------------------------------------------
%%
%% It may be distributed under the conditions of the LaTeX Project Public
%% License, either version 1.2 of this license or (at your option) any
%% later version.  The latest version of this license is in
%%    http://www.latex-project.org/lppl.txt
%% and version 1.2 or later is part of all distributions of LaTeX
%% version 1999/12/01 or later.
%%
%% The list of all files belonging to the 'Elsarticle Bundle' is
%% given in the file `manifest.txt'.
%%
%% Template article for Elsevier's document class `elsarticle'
%% with harvard style bibliographic references
%%
%% $Id: dib-template.tex 185 2020-08-07 09:06:08Z rishi $
%%
%% Use the option review to obtain double line spacing
%\documentclass[times,review,preprint]{elsarticle}

%% Use the options `final' to obtain the final layout
%% Use longtitle option to break abstract to multiple pages if overfull.
%% For Review pdf (With double line spacing)
%\documentclass[times,review]{elsarticle}
%% For abstracts longer than one page.
%\documentclass[times,review,longtitle]{elsarticle}
%% For Review pdf without preprint line
%\documentclass[times,review,nopreprintline]{elsarticle}
%% Final pdf
\documentclass[times,final]{elsarticle}
%%
%\documentclass[times,final,longtitle]{elsarticle}
%%

%%
%% Stylefile to load DIB template
\usepackage{dib}
\usepackage{framed,multirow}

%% The amssymb package provides various useful mathematical symbols
\usepackage{amssymb}
\usepackage{latexsym}

%% For line numbers
%\usepackage[switch]{lineno}

% Following three lines are needed for this document.
% If you are not loading colors or url, then these are
% not required.
\usepackage{url}
\usepackage{xcolor}
\definecolor{newcolor}{rgb}{.8,.349,.1}

%%
\usepackage{longtable}
\usepackage[colorlinks]{hyperref}

\journal{Data in Brief}

\begin{document}

\verso{Given-name Surname \textit{etal}}

\begin{frontmatter}

\dochead{Data Article}
%The article title must include the word 'data' or 'dataset'.
%Please avoid the use of acronyms and abbreviations where possible.
%For co-submission authors, the title should be unique,
%i.e. not the same as your research paper.
%A maximum of 250 characters is allowed.
\title{Article Title\tnoteref{tnote1}}%
\tnotetext[tnote1]{This is an example for title footnote coding.}
%Tip: here are a few examples of recent suitable article titles - these are short and clear:
%%Adolescent Rat Social Play: Amygdalar Proteomic and Transcriptomic Data
%%Execution Data Logs of a Supercomputer Workload Over its Extended Lifetime
%%Calgary Preschool Magnetic Resonance Imaging (MRI) Dataset]

%%Authors
\author[1]{Given-name1 \snm{Surname1}\corref{cor1}}
\cortext[cor1]{Corresponding author:
  Tel.: +0-000-000-0000;
  fax: +0-000-000-0000;}
\author[1]{Given-name2 \snm{Surname2}\fnref{fn1}}
\fntext[fn1]{This is author footnote for second author.}
\author[2]{Given-name3 \snm{Surname3}}
%% Third author's email
\ead{author3@author.com}
\author[2]{Given-name4 \snm{Surname4}}

%%Affiliations
\address[1]{Affiliation 1, Address, City and Postal Code, Country}
\address[2]{Affiliation 2, Address, City and Postal Code, Country}

%\received{1 May 2013}
%\finalform{10 May 2013}
%\accepted{13 May 2013}
%\availableonline{15 May 2013}
%\communicated{S. Sarkar}


\begin{abstract}
%%%%
Please Type your abstract here.

\noindent[The Abstract should describe the data collection process, the analysis
performed, the data, and their reuse potential. It should not provide
conclusions or interpretive insights. If your article is being
submitted via another Elsevier journal as a co-submission, please cite
this research article in the abstract.

\noindent\textbf{Tip:} do not use words such as
`study', `results', and `conclusions' because a data article should be
describing your data only.  Min 100 words - Max 500 words.]
%%%%
\end{abstract}

\begin{keyword}
%% Keywords
%[Include 4-8 keywords (or phrases) to facilitate others finding your
%article online.
%\noindent\textbf{Tip:} Try Google Scholar to find which terms are most common in your
%field. In biomedical fields, MeSH terms are a good 'common vocabulary'
%to draw from]
\KWD Keyword1\sep Keyword2\sep Keyword3
\end{keyword}

\end{frontmatter}

%% For linenumbers
%\linenumbers
\section*{Data in Brief Article Template}
%% main text
\noindent \href{https://www.journals.elsevier.com/data-in-brief/about-data-in-brief/data-in-brief-faq}%
{\textit{Data in Brief}}
is an open access journal that publishes data articles.
Please note:
\begin{itemize}
\item A data article is different to a research article, so it is
important to \textbf{use the template} below to prepare your manuscript for Data
in Brief.
\item A data article should \textbf{simply describe data} without providing
conclusions or interpretive insights.
\item Before you start writing your data article you should read the
guidance on
\href{https://www.journals.elsevier.com/data-in-brief/policies-and-guidelines/what-data-are-suitable-for-data-in-brief}%
{What Data are Suitable for Data in Brief}.
\item It is mandatory that Data in Brief authors share their research data:
\begin{itemize}
\item If you have \textbf{raw data} (also referred to as primary, source or
unprocessed data) relating to any charts, graphs or figures in the
manuscript, these data must be publicly available, either with the data
article (e.g. as a supplementary file) or hosted on a trusted data
repository.

\item If you are describing \textbf{secondary data} you are required to provide
a list of the primary data sources used \underline{and} to make the full secondary
dataset publicly available, either with the data article (e.g. as a
supplementary file) or hosted on a trusted data repository.

\item Although we allow supplementary files, it is preferred that
authors deposit their data in a trusted data repository ($>$70\% of
Data in Brief authors now do this). See our?
\href{https://www.elsevier.com/authors/author-resources/research-data/data-base-linking#repositories}%
{list of supported data repositories}.

\item For data that, for ethical reasons, require access controls a
mechanism must be provided so that our Editors and reviewers may access
these data without revealing their identities to authors (more
information is provided in the template \hyperlink{target1}{below}).
\end{itemize}
\end{itemize}

Have you any questions? See a list of frequently asked questions
\href{https://www.journals.elsevier.com/data-in-brief/about-data-in-brief/data-in-brief-faq}{here},
or email our Managing Editors:
\href{mailto:dib-me@elsevier.com}{dib-me@elsevier.com}. This
step-by-step?
\href{https://www.journals.elsevier.com/data-in-brief/about-data-in-brief/how-to-submit-your-research-data-article-data-in-brief}%
{video}?guide will also tell you how to complete the
template correctly to maximise your chances of acceptance.

\vskip6pt
\noindent Authors can submit to Data in Brief in two ways:

\begin{enumerate}
\item[\bf(1)] \textbf{If you are submitting your data article directly to Data in
Brief, you can now skip the next section and complete the}
\hyperlink{target2}{\textbf{Data Article template}}.

\item[\bf(2)] \textbf{If you are submitting your data article to Data in Brief via
another Elsevier journal as a co-submission (i.e. with a Research
Article), please read the} \hyperlink{target3}{\textbf{Co-submission Instructions}}
\textbf{on the next page
before completing the} \hyperlink{target2}{\textbf{Data Article template}}.

\end{enumerate}

\hypertarget{target3}{}
\section*{Co-submission Instructions}
A co-submission to~\textit{Data in Brief}~is done at the same time that you
submit (or resubmit, after revision) a research article to another
Elsevier journal. For co-submissions you therefore submit your
\textit{Data in Brief} data article manuscript via the other journal's
submission system and \underline{not} directly to
\textit{Data in Brief} itself.

\vskip6pt\noindent
The other Elsevier journal's Guide for Authors will state if a
co-submission is offered by that journal, and any revision letter/email
you receive from a participating journal will contain an offer to
submit a data article to \textit{Data in Brief}.

\vskip6pt\noindent
\textbf{To complete a co-submission you will need to zip your
~\textit{Data in Brief}~
manuscript file and all other files relevant to the
~\textit{Data in Brief}~
submission (including any supplementary data files) into a single .zip
file, and upload this as a "Data in Brief"-labelled item in the other
journal's submission system when you submit manuscript to that journal.
The .zip file will then be automatically transferred to
~\textit{Data in Brief}~
when your research article is accepted for publication in the other
journal, and when published your original research article and data
article will link to each other on ScienceDirect.}

\vskip6pt\noindent
\textbf{As~\textit{Data in Brief}~is open access, a moderate article publication charge
(APC) fee is payable on publication. For more information about the
APC, please see
\href{https://www.elsevier.com/journals/data-in-brief/2352-3409/open-access-journal}%
{here}.}

\vskip6pt\noindent
\textit{Data in Brief} \underline{requires} that authors share their research data. This can
be done by submitting it with the data article (e.g. as a supplementary
file) or by hosting on a trusted data repository (the latter is
preferred). Failure to do this will delay publication of your
co-submission.

\vskip6pt\noindent
\textbf{If you have any questions, please contact:
\href{mailto:DIB@Elsevier.com}{DIB@Elsevier.com}}

\vskip6pt\noindent
Please note, authors should not republish the same data presented in
their original research article in a \textit{Data in Brief} co-submission, as
this could constitute duplicate publication; however, \textit{Data in Brief}
welcomes the publication of any data article that fulfils one or more
of the following criteria:

\checkmark A description of the supplementary data that would
previously have been hosted as supplementary electronic files alongside
your original research article.*

\checkmark A description of the full dataset or additional information
that will aid reuse of the data.

\checkmark A detailed description of the raw data relating to the
charts, graphs or figures in your companion research article, if making
these data available will substantially enhance reproducibility and/or
reanalysis of the data.

\checkmark Any negative datasets or data from intermediate experiments
related to your research.

\textsf{X} Review articles or supplemental files from a review article
are not considered original data and are typically unsuitable for Data
in Brief.

\vskip12pt\noindent
* If describing supplementary data that you previously planned to
publish as supplem
entary electronic files hosted alongside the original
research article, it is requested that you either\break
deposit these in a
repository (preferred) or submit these to \textit{Data in Brief} alongside the
data article. \textbf{They should not be published as supplementary files with
your research article in the other journal}.

\clearpage
\hypertarget{target2}{}
\section*{Data Article template}
\noindent
Please fill in the template below. All sections are mandatory unless
otherwise indicated. Please read all instructions in [square brackets]
carefully and ensure that you delete all instruction text (including
the questions) from the template before submitting your article.

\vskip6pt\noindent
Reminder: A data article simply describes data and should not provide
conclusions or interpretive insights, so \textbf{avoid} using words such as
`study', `results' and `conclusions'.

\vskip6pt\noindent
We would welcome feedback on this template and how it might be
improved. To provide anonymous feedback via a very short survey, please
click \href{https://forms.office.com/Pages/ResponsePage.aspx?id=P-50kiWUCUGif5-xXBBnXTeXkbO343VFrbpYVBvxdZtUM05UVjIwM0U4WlRKUldCOTNMRUQwOVRHTy4u}%
{here}.

\vskip6pt\noindent
{\small\textbf{\textit{Please delete this line and everything above it before submitting your
article, in addition to anything in [square brackets] below, including
in the Specifications Table}}\vskip6pt\hrule\vskip12pt}

{\fontsize{7.5pt}{9pt}\selectfont
%%%
\noindent\textbf{Specifications Table}

Every section of this table is mandatory.
Please enter information in the right-hand column and remove all the instructions
\begin{longtable}{|p{33mm}|p{94mm}|}
\hline
\endhead
\hline
\endfoot
Subject                & [Please select one CATEGORY for your manuscript from the list
					               available at:\break
                         \href{https://www.elsevier.com/__data/assets/excel_doc/0012/736977/%
                               DIB-categories.xlsx}{DIB categories}.]\\
\hline
Specific subject area  & [Briefly describe the narrower subject area. Max 150 characters]\\
\hline
Type of data           & [List the type(s) of data this article describes.
                         Simply delete from this list as appropriate:]
                         Table\newline
                         Image\newline
                         Chart\newline
                         Graph\newline
                         Figure\newline
                         [Any other type not listed- please specify]\\
%\clearpage
How data were acquired & [State how the data were acquired: E.g. Microscope,
                         SEM, NMR, mass spectrometry, survey* etc.\newline
                         Instruments: E.g. hardware, software, program\newline
                         Make and model and of the instruments used:\newline

                         {\fontsize{7pt}{8pt}\selectfont
                         *\,if you conducted a survey you must submit a copy of the
                         survey(s) used (either provide these as supplementary material
                         file or provide a URL link to the survey
                         in this section of the table).
                         If the survey is not written in English,
                         please provide an English-language translation.}]\\
\hline
Data format            & [List your data format(s). Note, unless you are describing secondary data,
                         all raw data must be provided (either with this data article or linked to a repository).
                         Simply delete from this list as appropriate:]\newline
                         Raw\newline
                         Analyzed\newline
                         Filtered\newline
                         [Any other format not listed- please specify]\\
\hline
Parameters for
data\newline
collection             & [Provide a brief description of which conditions were considered
                         for data collection. Max 400 characters]\\

\hline
Description of
data\newline
collection             & [Provide a brief description of how these data were collected.
                         Max 600 characters]\\
\hline
Data source location   & [Fill in the information available, and delete from this list as appropriate:\newline

                         Institution:\newline
                         City/Town/Region:\newline
                         Country:\newline
                         Latitude and longitude (and GPS coordinates, if possible) for collected samples/data:\newline


                         If you are describing secondary data, you are required to provide a list of
                         the primary data sources used in the section.\newline

                         Primary data sources:  ]\\
\hline
\hypertarget{target1}
{Data accessibility}   & [State here if the data are either hosted `With the article' or on a public repository.
                         In the interests of openly sharing data we recommend hosting your data in a
                         trusted repository ($>$70\% of Data in Brief authors now use a data repository).
                         See our \href{https://www.elsevier.com/authors/author-resources/research-data/data-base-linking#repositories}{list of supported data repositories}.
                         We suggest \href{https://data.mendeley.com/}{Mendeley Data} if you do not have a trusted repository.\newline

                         Please delete or complete as appropriate, either:]\newline

                         With the article\newline

                         [Or, if in a public repository:]\newline

                         Repository name: [Name repository]\newline
                         Data identification number: [provide number]\newline
                         Direct URL to data: [e.g. https://www.data.edu.com - please note,\newline
                         this URL should be working at the time of submission]\newline

                         [\textbf{In addition, for data with access controls only:} For data that,
                         for ethical reasons (i.e. human patient data),
                         require access controls please describe
                         how readers can request access these data and provide a link to any
                         Data Use Agreement (DUA) or upload a copy as a supplementary file.]\newline

                         Instructions for accessing these data:\newline

                         [\textit{Important: if your data have access controls a mechanism must also be
                         provided so that our Editors and reviewers may access these data
                         without revealing their identities to authors, please include
                         these instructions with your submission. Please contact the Managing
                         Editors (DIB-ME@DIB.com) if you have any questions.}]\\
\hline
Related
research\newline
article                & [If your data article is related to a research article - \textbf{especially
                         if it is a co-submission} - please cite your associated research
                         article here. Authors should only list \textbf{one article}.\newline

                         Authors' names\newline
                         Title\newline
                         Journal\newline
                         DOI: \textbf{OR} for co-submission manuscripts `In Press'\newline

                         \textbf{For example, for a direct submission:}\newline

                         J. van der Geer, J.A.J. Hanraads, R.A. Lupton, The art of writing a scientific article,
                         J. Sci. Commun. 163 (2010) 51-59. https://doi.org/10.1016/j.Sc.2010.00372\newline

                         \textbf{Or, for a co-submission (when your related research article has not yet published):}\newline

                         J. van der Geer, J.A.J. Hanraads, R.A. Lupton, The art of writing a
                         scientific article, J. Sci. Commun. In Press.\newline

                         \textbf{Or, if your data article is not directly related to a research article,
                         please delete this last row of the table.}]
\end{longtable}
}
%%%

\section*{Value of the Data}

[Provide 3-6 bullet points explaining why these data are of value to the scientific community.
Bullet points 1-3 must specifically answer the questions next to the bullet point,
but do not include the question itself in your answer. You may
provide up to three additional bullet points to outline the value of these data.
Please keep points brief, with ideally no more than 400 characters for each point.]

\begin{itemize}
\itemsep=0pt
\parsep=0pt
\item Your first bullet point must explain why these data are useful or important?
\item Your second bullet point must explain who can benefit from these data?
\item Your third point bullet must explain how these data might be used/reused for
further insights and/or development of experiments.
\item In the next three points you may like to explain how these data could
potentially make an impact on society and highlight any other additional value of these data.
\item ....
\end{itemize}

\section*{Data Description}

\noindent [Individually describe each data file (i.e. figure 1, figure 2, table
1, dataset, raw data, supplementary data, etc.) that are included in
this article. Please make sure you refer to every data file and provide
a clear description for each - do not simply list them. No insight,
interpretation, background or conclusions should be included in this
section. Please include legends with any tables, figures or graphs.

\noindent\textbf{Tip:} do not forget to describe any supplementary data files.]

\section*{Experimental Design, Materials and Methods}

\noindent [Offer a complete description of the experimental design and methods
used to acquire these data. Please provide any programs or code files
used for filtering and analyzing these data. It is very important that
this section is as comprehensive as possible. If you are submitting via
another Elsevier journal (a co-submission) you are encouraged to
provide more detail than in your accompanying research article. There
is no character limit for this section; however, no insight,
interpretation, or background should be included in this section.

\noindent\textbf{Tip:} do not describe your data (figures, tables, etc.) in this section,
do this in the Data Description section above.]

\section*{Ethics Statement}
\noindent [Please refer to the journal's
\href{https://www.elsevier.com/journals/data-in-brief/2352-3409/guide-for-authors}{Guide for Authors}
for more information on
the ethical requirements for publication in Data in Brief. In addition
to these requirements:

\noindent\textbf{If the work involved the use of human subjects:}
please include a statement here confirming that informed consent was
obtained for experimentation with human subjects;

\noindent\textbf{If the work involved animal experiments:} please
include a statement confirming that all experiments comply with
the \href{https://www.nc3rs.org.uk/arrive-guidelines}{ARRIVE\ guidelines} and were be carried out in accordance with the
U.K. Animals (Scientific Procedures) Act, 1986 and associated
guidelines, \href{https://ec.europa.eu/environment/chemicals/lab_animals/legislation_en.htm}{EU Directive 2010/63/EU for animal experiments}, or the
National Institutes of Health guide for the care and use of Laboratory
animals (NIH Publications No. 8023, revised 1978)]

\section*{Acknowledgments}
Acknowledgments should be inserted at the end of the paper, before the
references, not as a footnote to the title. Use the unnumbered
Acknowledgements Head style for the Acknowledgments heading.

\section*{Declaration of Competing Interest}

\noindent [All authors are required to report the following information:
\begin{enumerate}
\item[(1)] All third-party financial support for the work this article;

\item[(2)] All financial relationships with any entity that could be
viewed as relevant to data described in this manuscript;

\item[(3)] All sources of revenue with relevance to this work where
payments have been made to authors, or their institutions on their
behalf, within the 36 months prior to submission;

\item[(4)] Any other interactions with the sponsor, outside of the
submitted work;

\item[(5)] Any relevant patents or copyrights (planned, pending or
issued);

\item[(6)] Any other relationships or affiliations that may be
perceived by readers to have influenced, or give the appearance of
potentially influencing, what has been written in this article.
\end{enumerate}
As a general guideline, it is usually better to
disclose a relationship than not. This information will be acknowledged
at publication in the manuscript. If there is no known competing
financial interests or personal relationships that could have appeared
to influence the work reported in this paper, please include this
statement.]

\vskip12pt\noindent
The authors declare that they have no known competing
financial interests or personal relationships which have, or could be
perceived to have, influenced the work reported in this article.

\vskip12pt\noindent
[If there are financial interests/personal relationships which may be
considered as potential competing interests, please declare them here.]

\subsection*{Note}
\label{sec1}
Any instructions relevant to the \verb+elsarticle.cls+ are applicable
here as well. See the online instruction available on:
\makeatletter
\if@twocolumn
\begin{verbatim}
 http://support.river-valley.com/wiki/
 index.php?title=Elsarticle.cls
\end{verbatim}
\else
\begin{verbatim}
 http://support.river-valley.com/wiki/index.php?title=Elsarticle.cls
\end{verbatim}
\fi

\section*{References}

\noindent [References are limited (approx. 15) and excessive self-citation is not
allowed. \textbf{If your data article is co-submitted via another Elsevier
journal, please cite your associated research article here.}

\noindent\textbf{Reference style:}
Text:?Indicate references by number(s) in square brackets in line with
the text. The actual authors can be referred to, but the reference
number(s) must always be given.?

\noindent Example: '..... as demonstrated [3,6]. Barnaby and Jones [8] obtained a different result ....'?

\noindent [Use \verb+\cite+ command to cite a reference list item in text.

\noindent These are examples for reference citations \cite{1}.
\cite{2}.
\cite{4}.]

\subsection*{Reference list using Bib\TeX database file}
\noindent [If Bib\TeX database file is used for reference data please use
\begin{verbatim}
\bibliographystyle{model1-num-names}
\bibliography{<BibTeX file name>}
\end{verbatim}
and use bibtex command to generate list of references.]

%% Numbered
%%If
\bibliographystyle{model1-num-names}
\bibliography{refs}

\subsection*{Reference list using bibliography envrironment}

\noindent [List:?Number the references (numbers in square brackets) in
the list in the order in which they appear in the text.?

\noindent Examples:?

{\small
\begin{verbatim}
\begin{thebibliography}{0}
\bibitem{1} J. van der Geer, J.A.J. Hanraads, R.A. Lupton, The art of
writing a scientific article, J. Sci. Commun. 163 (2010) 51-59.
https://doi.org/10.1016/j.Sc.2010.00372.
\bibitem{2} Van der Geer, J., Hanraads, J.A.J., Lupton, R.A., 2018. The
art of writing a scientific article. Heliyon. 19, e00205.
https://doi.org/10.1016/j.heliyon.2018.e00205.
\bibitem{3} W. Strunk Jr., E.B. White, The Elements of Style, fourth ed.,
Longman, New York, 2000.?
\bibitem{4} G.R. Mettam, L.B. Adams, How to prepare an electronic
version of your article, in: B.S. Jones, R.Z. Smith (Eds.),
Introduction to the Electronic Age, E-Publishing Inc., New York, 2009,
pp. 281-304.
\bibitem{5} Cancer Research UK, Cancer statistics reports for the UK.
http://www.cancerresearchuk.org/aboutcancer/statistics/cancerstatsreport/,
2003 (accessed 13 March 2003).
\bibitem{6} [dataset] M. Oguro, S. Imahiro, S. Saito, T. Nakashizuka,
Mortality data for Japanese oak wilt disease and surrounding forest
compositions, Mendeley Data, v1, 2015.
https://doi.org/10.17632/xwj98nb39r.1.
\end{thebibliography}
\end{verbatim}
}

\noindent Example:]
\vspace*{-12pt}

\begin{thebibliography}{0}
\item[] \textbf{Reference to a journal publication:}
\bibitem{1} J. van der Geer, J.A.J. Hanraads, R.A. Lupton, The art of
writing a scientific article, J. Sci. Commun. 163 (2010) 51-59.
https://doi.org/10.1016/j.Sc.2010.00372.

\textbf{Reference to a journal publication with an article number:}
\bibitem{2} Van der Geer, J., Hanraads, J.A.J., Lupton, R.A., 2018. The
art of writing a scientific article. Heliyon. 19, e00205.
https://doi.org/10.1016/j.heliyon.2018.e00205.

\textbf{Reference to a book:}
\bibitem{3} W. Strunk Jr., E.B. White, The Elements of Style, fourth ed., Longman, New York, 2000.?

\textbf{Reference to a chapter in an edited book:}
\bibitem{4} G.R. Mettam, L.B. Adams, How to prepare an electronic
version of your article, in: B.S. Jones, R.Z. Smith (Eds.),
Introduction to the Electronic Age, E-Publishing Inc., New York, 2009,
pp. 281-304.

\textbf{Reference to a website:}
\bibitem{5} Cancer Research UK, Cancer statistics reports for the UK.\newline
http://www.cancerresearchuk.org/aboutcancer/statistics/cancerstatsreport/, 2003 (accessed 13 March 2003).

\textbf{Reference to a dataset:}
\bibitem{6} [dataset] M. Oguro, S. Imahiro, S. Saito, T. Nakashizuka,
Mortality data for Japanese oak wilt disease and surrounding forest
compositions, Mendeley Data, v1, 2015.
https://doi.org/10.17632/xwj98nb39r.1.]
\end{thebibliography}

\end{document}

%%
